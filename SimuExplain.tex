\documentclass[français,final,twoside,12pt]{article}
\usepackage[utf8]{inputenc}
\usepackage[francais]{babel}
\usepackage[T1]{fontenc}
\usepackage{amsmath}
\usepackage{url}
\usepackage{graphicx}
\usepackage{palatino}
\usepackage{vmargin}
\usepackage{fancyhdr} 
\usepackage{array}
\usepackage{lmodern}
\usepackage{mathtools}


\setpapersize{A4}
\setmargrb{2cm}{1.5cm}{2cm}{1.5cm}

\title{Geant 4 DAMIC Simualtion: How to use it}
\author{LPNHE DAMIC}
\date{20/03/2017}

\begin{document}

\maketitle

\section{Running the simulation}
\subsection{Install Geant4}

To use the simulation you need to install Geant4. The version used is the Geant4 10.2 .  You can use newer versions. 

During installation of Geant 4 you should set to ON the option "GEANT4 INSTALL DATA". I recommend you to use QT for visualisation.  

\subsection{Run the simulation}


Once Geant4 installation is completed, go to the bin directory on your installation directory. Then use one of the bash files ( geant4.sh or geant4.csh). It will dowload the path for different datasets on your terminal. (You have to do it for each terminal where you want to run the simulation). \\


You need to create two directories. One of them will contain all the simulation source code (/SourceDirectory). The other one will be for compiling the code (/RunDirectory). \\

On your terminal open the empty directory (/RunDirectory) and compile the source code with the command : \textit{cmake \-DGeant4\_DIR=/PathtoyourG4InstallDirectory /pathtothe \_SourceDirectory}.
Then you just have to use the command : \textit{make install}.\\

If you change something on the code source you have to use the make install command to compile again. 
Finally, you could run the simulation.\\

If you want to use the visualization just call the executable on your terminal. It should open the QT interface. Then use the command \textit{/command/execute vis.mac}. The DAMIC Setup coded on Geant4 should appear. You can then use Geant4 commands on QT browser to visualize interaction.\\

If you only want to use the bash mode, you call the executable followed by the macro file that you want to use( file with extension \textit{.mac}

\section{Create a macro file for the simulation}

\subsection{Macrofile Skeleton}

We can divide the commands of a macro file in two parts :

\begin{itemize}
\item A Geant4 part where you put some information about the Geant4 simulation and two mandatory commands.
\item A part where you choose your primary generator ( Particle type, Energy, Momentum...),and  where to generate it  (Volume, Shape, Source..). The commands of this part are made only for the DAMIC simulation only.
\end{itemize}
 
\subsection{Geant4 Commands}

There is only two mandatory commands :
\begin{itemize}
\item /run/initialize
\item /run/beamOn N
\end{itemize}




N is the number of primary particles that you want to simulate.For other Geant4 commands take a look at the Geant4 documentation. Don't use the Gun Particle commands of Geant4 they will not work. 

\subsection{DAMIC Commands}

At the moment the DAMIC Commands serves to parameterize the DAMIC Particle Gun. The Particle Gun is a code that defines the primary particle and its properties (also named primary generator).\\ 

The DAMIC  Commands are divided in 4 categories:
\begin{itemize}
\item Particle  : Commands that change the primary generator definition
\item Energy  : Commands that parametrized the primary generator energy
\item Direction : Commands that parametrized the primary generator momentum direction
\item Position : Commands that parametrized the primary generator position
\newline
\end{itemize}

The DAMIC commands always begin with \textit{/damic/gun/}

\subsubsection{DAMIC Particle Commands}

There is 4 "Particle Commands" : 
\\
\begin{itemize}
\item /damic/gun/list : This command gives the list of the particles that can be define as primary generators. 
\item /damic/gun/particle nameparticle : Define the primary generator particle ( mass, charge...). If you want to define an ion ( for decay process) the nameparticle should be replaced by ion.
\item /damic/gun/charge :  Change the primary generator charge, if different from the standard one.
\item /damic/gun/ion Z A Q E : Defines the ion used for decay. Z is the atomic number, A is the AtomicMass, Q is the Charge of the ion, E the excitation energy. You need to use the command \textit{/damic/gun/particle ion} before using this one. 
\end{itemize}
\\

\subsubsection{DAMIC energy commands}
There is 4 "Energy Commands", three of them are dependant:\\

\begin{itemize}
\item /damic/gun/energy/mono Energy Unit : Set a monoenergetic energy of \textit{Energy Unit} for the primary generator (Energy Units).
\item /damic/gun/energy/distri Candidate:  Set a distribution energy \textit{Candidate} for the primary generator. Candidates : Uniform.
\item /damic/gun/energy/bot Energy Unit: Set the bottom energy value for a uniform energy distribution. 
\item /damic/gun/energy/top Energy Unit: Set the top energy value for a uniform energy distribution.
\end{itemize}
\\

\subsubsection{DAMIC direction commands}
There is 9 "Direction Commands", divided in two parts. One to define a single particle generator direction and the second to define a distribution for the particle generator direction.\\

The 4 commands to define a single direction:\\
\begin{itemize}
\item /damic/gun/direction/oned : Set the direction distribution to a single direction.
\item /damic/gun/direction/onedX : Set the x value of the direction.
\item /damic/gun/direction/onedY : Set the y value of the direction.
\item /damic/gun/direction/onedZ : Set the z value of the direction.
\newline
\end{itemize}


The 5 commands to define a distribution for the particle generator direction ( Only Isotropic for the moment):
\begin{itemize}
\item /damic/gun/direction/distri Candidate:  Set a distribution direction for the primary generator. Candidates: Isotropic.
\item /damic/gun/direction/thetamin Value: Set the minimum value of theta for an isotropic distribution
\item /damic/gun/direction/thetamax Value: Set the maximum value of theta for an isotropic distribution
\item /damic/gun/direction/phimin Value: Set the minimum value of phi for an isotropic distribution
\item /damic/gun/direction/phimax Value: Set the maximum value of phi for an isotropic distribution
\end{itemize}

\subsubsection{DAMIC position commands}

There is  a lot of commands for the position. There is four ways to define the position of the primary generator. You can generate the primary particle:
\begin{itemize}
\item from a point source
\item following a shape (Parallelepiped, Cylinder, Sphere)
\item from a specific material
\item from different parts of the DAMIC geometry
\newline
\end{itemize}



To generate from a point source: 

\begin{itemize}
\item /damic/gun/position/dosource: Set the position to a point source
\item /damic/gun/position/centerx Value Unit: Set the X coordinate of the point source 
\item /damic/gun/position/centery Value Unit: Set the Y coordinate of the point source 
\item /damic/gun/position/centerz Value Unit: Set the Z coordinate of the point source
\newline 
\end{itemize}

To generate following a shape:

\begin{itemize}
\item /damic/gun/position/doshape shape: Set the position to a shape. Candidates: Para Sphere Cyl
\item /damic/gun/position/centerx Value Unit: Set the X coordinate of the shape center
\item /damic/gun/position/centery Value Unit: Set the Y coordinate of the shape center 
\item /damic/gun/position/centerz Value Unit: Set the Z coordinate of the shape center
\item /damic/gun/position/parax Value Unit : Set the lenght of the X side of the parallelepiped distribution.
\item /damic/gun/position/paray Value Unit : Set the lenght of the Y side of the parallelepiped distribution.
\item /damic/gun/position/paraz Value Unit : Set the lenght of the Z side of the parallelepiped distribution.
\item /damic/gun/position/spherer Value Unit: Set the lenght of the radius of the sphere distribution.
\item /damic/gun/position/cylr Value Unit: Set the lenght of the radius of the cylinder distribution.
\item /damic/gun/position/cylh Value Unit: Set the lenght of the height of the cylinder distribution
\newline
\end{itemize}



To generate from a material: 

\begin{itemize}
\item /damic/gun/position/domaterial Name: Set the position to a material source. Candidates: All materials of the DAMIC geometry. Look at DAMICDetectorConstruction.cc
\item /damic/gun/position/mother Name: Set the mother volume to generate primaries in the material chosen.
\newline 
\end{itemize}


To generate from different parts of the DAMIC geometry:

\begin{itemize}
\item /damic/gun/position/dovolume: Set the position to volumes.
\item /damic/gun/position/addvolume Name Activity: Add volumes where primaries will be generated. Acitivity is in $\mu Bq.kg^{-1}$ 
\newline
\end{itemize}

There is a Rootmacro that creates the macrofiles automaticaly for several volumes depending on a simple configuration file.  

\end{document}



